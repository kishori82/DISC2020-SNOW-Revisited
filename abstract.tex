\rots{} that read data distributed across servers dominate the workloads of real-world distributed storage systems.
The SNOW Theorem~\cite{SNOW2016} proved that ideal \rots{} that have optimal latency and the strongest guarantees---i.e., ``SNOW'' \rots{}---are impossible in one specific setting. That setting requires three or more clients, at least two readers among the clients (SWMR), and does not allow client-to-client communication.
It left many open questions around when SNOW is possible that depend on whether client-to-client communication is allowed, whether clients are readers or writers, and when there are two clients.
%It implicitly left open the possibility of SNOW with client-to-client communication, 
%implicitly left open the possibility of SNOW with one reader and multiple writers (MWSR), and
%explicitly left open the possibility of SNOW with two clients.
We close all of these open questions with new impossibility results or new algorithms.
We prove SNOW is impossible when client-to-client communication is allowed with three or more clients.
That proof uses two writers and one reader, and thus also shows SNOW is impossible for multiple writers and a single reader (MWSR) with client-to-client communication.
We show SNOW's possibility with two clients (SWSR) is dependent on client-to-client communication:
We give an algorithm that provides SNOW when client-to-client communication is allowed;
we prove it is impossible when client-to-client communication is disallowed.






% Large-scale web services are built on distributed storage systems that \textit{shard} data across many machines within a datacenter and \textit{geo-replicate} data across several datacenters.
% At the heart of designing such systems is the trade-off between the strength of guarantees provided to higher-level applications and the performance of the system.
% The performance-guarantee trade-off due to replication has been well studied with impossibility results such as FLP~\cite{Fischer:pds1983} and CAP~\cite{Brewer:pdc2000, Gilbert:sigact2002}. % with several well-known impossibility results that influence the design of practical systems.
% The trade-off due to sharding, in contrast, is far less understood with only one result that is relevant for the \rots{} that dominate real-world workloads.
% This result is the SNOW Theorem \cite{SNOW2016}, which proves the ideal \rots{} that have optimal latency and the strongest guarantees---i.e., ``SNOW'' \rots{}---are impossible.
% %This result is the SNOW Theorem, which proves that it is impossible for \rots{} to have all of the S, N, O, and W properties, where the SNOW properties represent the optimal latency and strongest guarantees.
% Our work builds on the SNOW Theorem to clarify this fundamental trade-off by closing several open questions.
% %and provide rigorously proved results.
% These questions relate to the possibility of SNOW with two clients, multiple writers and a single reader, and when client-to-client communication is allowed.
% We prove that SNOW is possible in some of these settings, but that it is impossible in the most general case even with client-to-client communication.
% %This general impossibility raises the question of what the best achievable latency is for \rots{} with the strongest guarantees.
% %Surprisingly, no existing algorithm provides bounded latency.
% %We present the first algorithms with bounded latency for \rots{} with the strongest guarantees.
% %We provide rigorous proofs for the SNOW theorem with three clients and show the impossibility of SNOW with two clients.
% Our proofs require delicate reasoning about the inter-play of the SNOW properties and asynchrony, therefore, we rely heavily on I/O automata theory to derive these results.

%Surprisingly, no existing algorithm provides bounded latency.
%We present two algorithms with bounded latency that differ in the number of rounds they take and the number of versions of data they return.
%Our final algorithm provides optimal latency in its best case and has a worst case of two non-blocking rounds that return two versions of the requested data.


%% In this paper, we revisit SNOW with greater rigor to strengthen the fundamental understandings in sharding dimension.
%% We present a rigor proof of the SNOW Theorem using I/O automata and five new (im)possibility findings with different system settings and client-to-client communication.
%% We also present a set of novel {\sc READ} transaction algorithms that achieve bounded latency in strictly serializable systems that have transactional writes.
%% Previous work either potentially requires indefinite blocking and/or unbounded number of round trips.



% Large-scale web services are built on distributed storage systems, that \textit{shard} (i.e., partition) data across many machines within a datacenter and \textit{geo-replicate} the data across several datacenters.
% %Their distributed nature within and across datacenters suggests a fundamental tradeoff between performance and consistency.
% At the heart of designing such systems is the trade-off between the strength of guarantees provided to overlying applications and the performance of the system.
% %The performance-guarantee trade-off due to replication has been well studied with impossibility results such as FLP~\cite{Fischer:pds1983} and CAP~\cite{Brewer:pdc2000, Gilbert:sigact2002}. % with several well-known impossibility results that influence the design of practical systems.
% The trade-off due to sharding, in contrast, is far less understood with only one result that is relevant for the \rots{} that dominate real-world workloads.
% This result is the SNOW Theorem \cite{SNOW2016} that proves the ideal \rots{} that have optimal latency and the strongest guarantees---i.e., ``SNOW'' \rots{}---are impossible.
% %This result is the SNOW Theorem, which proves that it is impossible for \rots{} to have all of the S, N, O, and W properties, where the SNOW properties represent the optimal latency and strongest guarantees.
% Our work builds on the SNOW Theorem to clarify this fundamental trade-off by closing several open questions and provide rigorously proved results.
% These questions relate to the possibility of SNOW with two clients, multiple writers and a single reader, and when client-to-client communication is allowed.
% We prove that SNOW is possible in some of these settings, but that it is impossible in the most general case even with client-to-client communication.
% %This general impossibility raises the question of what the best achievable latency is for \rots{} with the strongest guarantees.
% %Surprisingly, no existing algorithm provides bounded latency.
% We present the first algorithms with bounded latency for \rots{} with the strongest guarantees.
% We provide rigorous proofs for the SNOW theorem for the three-client  and show the impossibility of SNOW properties for transaction system with two-clients. Our proofs require delicate reasoning about the inter-play of the SNOW properties and asynchrony,therefore, we rely heavily on I/O automata theory to derive these results.

% %Surprisingly, no existing algorithm provides bounded latency.
% %We present two algorithms with bounded latency that differ in the number of rounds they take and the number of versions of data they return.
% %Our final algorithm provides optimal latency in its best case and has a worst case of two non-blocking rounds that return two versions of the requested data.


% %% In this paper, we revisit SNOW with greater rigor to strengthen the fundamental understandings in sharding dimension.
% %% We present a rigor proof of the SNOW Theorem using I/O automata and five new (im)possibility findings with different system settings and client-to-client communication.
% %% We also present a set of novel {\sc READ} transaction algorithms that achieve bounded latency in strictly serializable systems that have transactional writes.
% %% Previous work either potentially requires indefinite blocking and/or unbounded number of round trips.
