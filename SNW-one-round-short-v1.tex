%\section{\text{SN$\bar{o}$W} for MWMR setting}\label{mwmr_snow}
Here, we present  algorithm  $C$ which satisfies  SNW and "one-round"  properties in the MWMR setting, where a \textit{READ} consists one round of communications between the reader and the servers but servers may respond with multiple versions of the data. %We denote the type of object that satisfies the SN$\bar{o}$W properties by $\bar{\mathcal{O}}_T$.   
%Note that $B$ can also tolerate crash failure of writers and readers. 
The notation for the  writers, servers and tag are similar to the previous sub-section. 
The steps for the writers are shown in Pseudocode~\ref{fig:algo_bc} and  those for readers and the servers  are presented in Pseudocode~\ref{fig:algo_c}.
%
%We assume there is a set of writers $\mathcal{W}$,  a set of readers $\mathcal{R}$ and a set of $k \geq 1$ servers,  $\mathcal{S}$, with ids $s_1, s_2\cdots s_k$ that stores the objects $o_1, o_2, \cdots, o_k$, respectively.  We define  key ${\kappa}$  as a pair $(z, w)$, where $z \in \mathbb{N}$ and $w \in \mathcal{W}$ the  id of a writer. We use $\mathcal{K}$ to denote the set of all possible keys, which  are used to uniquely identify each transaction. Also, with each transaction we associate a tag $t \in \mathbb{N}$. 
%We assume writers can send message to the reader, i.e., we allow client-to-client messages.
% We assume that each process has a unique id. We assume that each server $s$ stores object $o_s$, due to sharding.  For the sake of simplicity, we assume that the \emph{WRITE} transactions issued by a writer are of the form: \emph{ write value $v_1$ to object $o_1$, $v_2$ to $o_2$, and so on}. In other words, we assume that such a  transaction consists of the set of operations $\{ \writeop{o_1}{v^{(1)}}, \writeop{o_2}{v^{(2)}}\cdots \writeop{o_k}{v^{(k)}}\} $, which we denote by $write(v_1, v_2, \cdots, v_k)$.  \emph{WRITE} transactions where only a subset of  the objects are updates will be discussed later. We also assume that  the \textit{READ} transactions are of the form  $\{ \readop{o_1}, \readop{o_2}\cdots \readop{o_k}\}$ consisting of $k$ reads to $k$ distinct objects.		
We designate one of the servers as the coordinator,  denote as $s^*$. 
%In addition to managing a shard object 
%object, the server $s^*$ is used to maintain the order of the \emph{WRITE}s and the objects that are updated during the \emph{WRITE} in the variable $List$. 
 % In any system, where there are many objects 
%different objects may be use different servers  as coordinators based on some load-balancing rule. 

\textit{\textbf{State variables:}} The state variables in algorithm $C$ is similar to those of algorithm $B$, therefore, we omit here.
			
\textit{\textbf{Writer steps:}} The steps of a \emph{WRITE} transaction is similarly to algorithm $B$, therefore, we omit here.


%At $s^*$, upon receiving  (\updateCoordTag, $({\kappa}, (b_{1}, \cdots, b_{k})$) from $w$ appends  
%			 $({\kappa}, (b_{1}, \cdots, b_{k}))$ to its  $List$,  and responds with  
%			 {\ackTag} and $t_{w}$ (set to be the number of elements in the local list $List$)  to $w$.
%			 The order of the  elements in  $List$ corresponds to  the order  
%the \emph{WRITE} transactions, the order of the incoming  {\updateCoordTag} updates,  as seen by $s^*$.
%			  After receiving an ({\ackTag}, $t_w$),  $w$ completes the \emph{WRITE}.

\textit{\textbf{Reader steps:}}  
The step  \Readtr{$ o_{i_1},  o_{i_2}, \cdots, o_{i_p}$} can be 
initiated by  some reader  $r$ intending to read the values of 
subset $o_{i_1},  o_{i_2}, \cdots, o_{i_p}$ of the objects. 
Denote  $I \triangleq \{i_1, i_2, \cdots, i_p\}$  and $S_I\triangleq \{s_{i_1}, s_{i_2}, \cdots, s_{i_p}\}$.
The procedure 
consists of only one  phase  of communication round
between the $r$ and the  servers, called  {\readValuesAndTags}. 
%
During  {\readValuesAndTags},  $r$ sends $s^*$ the message  {\getTagArrayTag}  
 requesting the  list of the latest added keys for each object, and also sends
    requests $(\text{\readValuesTag})$  each server $s_i$ in $S_I$. Note that if $s^*$ is also one of the 
    servers in $S_I$ then the {\getTagArrayTag} and  {\readValuesTag} messages to $s^*$ can be combined to create one message; however, we keep them separate for clarity of presentation.
% 
 Once $r$ receives a list of tags, such as, $(t_r, ({\kappa}_1, {\kappa}_2, \cdots,  {\kappa}_k))$ from $s^*$  and 
 the set of $Vals_{i}$ from each $s_i \in S_I$ then $r$ returns  the values 
 $v_{i_1}$, $v_{i_2}, \cdots v_{i_p}$ such that   $({\kappa}_{j}, v_{j}) \in Vals_{j}$, $j\in \{1, \cdots p\}$, and completes the 
\textit{READ}.
%
%

\textit{\textbf{Server steps:}}  
When a server $s_i$ receives a message $(${\writeValueTag}$, ({\kappa}, v_{i}))$ from a writer $w$ or 
 $s^*$, receives  (\updateCoordTag, $({\kappa}, (b_{1}, \cdots, b_{k})$) from writer $w$ or 
 receives  a message as  {\getTagArrayTag} $r$ the steps are similar to those in $B$. 
On the other hand, if any server $s_i$ receives a message  $(${\readValuesTag}$)$ from a reader $r$ then it responds to $r$ with  $Vals$. 
The following result states that $C$ respects SNW and ``one-round"  properties.
% and the proof is omitted for now since it is very straightforward. Note that the liveness property of \textit{READ} and \emph{WRITE}  transactions are a part of the SNOW properties.
%Consider any failure-free and fair execution of algorithm $C$. 
%For the purpose of proving the $S$ property, for every transaction transaction  $\phi$ in an execution of $C$ we associate a tag $tag(\phi)$ as described below.
% the variable $t_r$ in a reader and $t_w$ in 
%a writer are used to associate a tag as described below.
% $ \triangleq \max_{1 \leq j \leq |List|} \{ j : List[j].b_i = 1 \wedge i \in I\}$, which is presented as a comment in the pseudo-code for $A$.
%If $\phi$ is a \emph{WRITE} (\textit{READ}) then  $tag(\phi)$ is the value of the variable $t_w$ ($t_r$) immediately before the operation completes.

\begin{theorem} Any well-formed  and fair execution of  $C$, in the MWMR setting
  satisfies the SNW and "one-round" properties. %Algorithm $B$ also tolerates any client crash failures.
\end{theorem}
%\begin{theorem} Any well-formed  and fair execution of algorithm $C$ is an implementation of  an object of type $\bar{\mathcal{O}}_T$ in the MWMR setting, 	with no client-to-client communication,  comprising of objects $o_1, o_2, \cdots o_k$ stored in servers $s_1, s_2, \cdots, s_k$, respectively; and it satisfies the SN$\bar{o}$W  properties. %Algorithm $B$ also tolerates any client crash failures. \end{theorem}
%In algorihtm C, when a server $s_i$ responds to a reader with the entire history of  update values $Vals_i$ for object $o_i$. This impractical behavior 
%can be circuvented by sending only the $W$ latest values of $o_i$ to a read request, where $W$ is the maximum number of possible \emph{WRITE}s concurrent with any \textit{READ}.

\remove{
\begin{proof} Below we show that algorithm $C$ satisfies the  SN$\bar{o}$W properties. 
	
	\noindent{\emph{\underline{S property:}}} 
	Let $\beta$ be any fair execution  of  $B$ and 
 suppose all clients in $\beta$ behave in a well-formed
manner. Suppose $\beta$ contains no incomplete transactions and let  $\Pi$ be the set of transactions in $\beta$.  We define an irreflexive partial ordering ($\prec$) in $\Pi$ as follows:  if $\phi$ and $\pi$ are any two distinct transactions in $\Pi$ then we say 
	$\phi \prec \pi$ if either $(i)$ $tag(\phi) < tag(\pi)$ or $(ii)$ $tag(\phi) = tag(\pi)$ and $\phi$ is a \emph{WRITE} and $\pi$ is a \textit{READ}. Below we prove the $S$  property of $B$ by showing that  properties $P1$, $P2$, $P3$ and $P4$ of Lemma~\ref{lem:equivalence} hold for $\beta$.  The properties $P1$-$P4$ can be proved to hold in a manner very similar to algorithm $B$ (Section~\ref{mwmr_snow_one_version}). Therefore, we avoid repeating them. 

	\noindent{\emph{\underline{N, $\bar{o}$ and W properties:}}}  Evident from an inspection of the algorithm.
	% for the  response steps  of the servers to the reader.
	\end{proof}
}

