%\section{SNOW on MWSR with client-to-client messages} \label{mwsr}
\vspace{-1.3em}
\subsection{SNOW with C2C Communication}
 \label{subsec:yes_snow_yes_c2c}
\label{sec:mwsr}
In this section, we state that  SNOW is possible in the   \emph{multiple-writers single-reader} 
(MWSR) setting %, and prove that any fair and well-formed execution  of $A$ satisfies  the SNOW properties.  
%In practice, a system with a single reader may not be very useful but this algorithm serves a counter example 
%algorithm
 when client-to-client communication is allowed.  
%Algorithm $A$ 
%shows that if client-to-client communication is allowed, it is possible to have algorithms  that satisfies all of the SNOW properties with two clients. 
We consider a system that has $\ell \geq 1$ writers with ids $w_1, 
w_2 \cdots w_{\ell} \in \mathcal{W}$ 
%(we denote this set by $\mathcal{W}$)
, one reader $r$, and  $k \geq 1$ servers with ids $s_1, s_2\cdots s_k \in \mathcal{S}$. 
%(denote as $\mathcal{S}$) that maintains the  objects $o_1, \cdots, o_k$, respectively. 
Client-to-client communication is allowed. 
%Note that for a two-client system, when both  clients are of the same type, i.e., two writers or two readers, the SNOW properties are trivially satisfied.
	%			
% For any two tags $t_1, t_2 \in \mathcal{T}$ we say  $t_2 > t_1$ if $(i)$ $t_2.z > t_1.z$ or $(ii)$ $t_2.z = t_1.z$ and $t_2.w > t_1.w$. Therefore, the set $\mathcal{T}$ is totally ordered set. We also assume that every client and servers has a unique id and the ids can be compared w.r.t. some lexicographic order.
%
The following theorem states that in such a setting  it is possible to have an algorithm  that respects all SNOW properties. Due to space constraints, we do not present the algorithm or the proof for the theorem, which can be found in Appendix~\ref{app:algorithm-a}. %, where \wots{} are live. % and the proof is omitted, for now, since it is very straightforward. 


%Consider any failure-free execution of algorithm $A$. In the steps for the reader assume the quantity
 %$t_r \triangleq \max_{1 \leq j \leq |List|} \{ j : List[j].b_i = 1 \wedge i \in I\}$, which is presented as a comment in the pseudo-code for $A$.
%We associate with any transaction $\phi$ a tag
%$tag(\phi)$ such that if  $\phi$ is a \wot{}  $tag(\phi)=t_w$, i.e., the value of $t_w$ before the completion of the operation, and $tag(\phi)=t_r$ when $\phi$ is a \rot{}.
		
\begin{theorem} In the MWSR setting, there exists an algorithm $A$ such that
	any  well-formed  and fair execution of $A$  implements a wait-free  transaction processing system, $T$, for objects of type $\mathcal{O}_T$, consisting of  objects $o_1, o_2, \cdots o_k$ at servers $s_1, s_2, \cdots, s_k$, respectively; and $T$ respects the SNOW properties and \wots{} are live.
\end{theorem}
	
\remove{
	\begin{proof} Below we show that $A$ satisfies the  SNOW properties. 
	
	\noindent{\emph{\underline{S property:}}} 
	Let $\beta$ be any fair execution  of  $A$ and 
 suppose all clients in $\beta$ behave in a well-formed
manner. Suppose $\beta$ contains no incomplete transactions and let  $\Pi$ be the set of transactions in $\beta$.  We define an irreflexive partial ordering ($\prec$) among the transactions in $\Pi$ as follows:  if $\phi$ and $\pi$ are any two distinct transactions in $\Pi$ then we say 
	$\phi \prec \pi$ if either $(i)$ $tag(\phi) < tag(\pi)$ or $(ii)$ $tag(\phi) = tag(\pi)$ and $\phi$ is a \wot{} and $\pi$ is a \rot{}. We will prove the $S$  (strict-serializability) property of $A$ by proving that the properties $P1$, $P2$, $P3$ and $P4$ of Lemma~\ref{lem:equivalence} hold for $\beta$. 
	
	\emph{P1:}   If $\pi$ is a \rot{} then since all \rots{} are invoked by a single reader $r$ and in a well-formed manner, 
	therefore, there cannot be an infinite number of \rots{} such that they all 
	precede $\pi$ (w.r.t $\prec$).
	 Now, suppose $\pi$ is a \wot{}. Clearly, from an inspection of the algorithm, 
	 $tag(\pi) \in \mathbb{N}$. From inspection of the algorithm, each \wot{} increases the size of 
	 $List$, and the value of the tags are  defined by the size of $List$. Therefore, there can be at 
	 most a finite number of \wots{} such that can precede $\pi$ (w.r.t. $\prec$) in $\beta$.
	  
	\emph{P2:}  Suppose $\phi$ and $ \pi$ are any two transactions in $\Pi$, such that, $\pi$ begins after $\phi$ completes. 
	Then we show that we cannot have $\pi \prec \phi$. Now, we consider four cases, depending on whether $\phi$ and $\pi$ are \rots{} or \wots{}.	
	\begin{enumerate}
	    \item [$(a)$] $\phi$ and $\pi$ are \wots{} invoked by writers $w_{\phi}$ and $w_{\pi}$, respectively. Since the size of $List$, in $r$,  grows monotonically with each \wot{}  hence  $w_{\pi}$ receives the  tag at least as high as $tag(\phi)$, so $\pi\not \prec \phi$.
	      %
	       \item [$(b)$] $\phi$ is a \wot{}, $\pi$ is a \rots{} invoked by writer $w_{\phi}$ and $r$, respectively.  
	        Since the size of $List$, in $r$,  grows monotonically, and because  $w_{\pi}$ invokes $\pi$ after $\phi$ completes hence  $tag(\pi)$ is at least as high as $tag(\phi)$, so $\pi\not \prec \phi$.
	      %
	        \item[$(c)$] $\phi$ and $\pi$ are \rots{}  invoked by reader $r$. 
	           Since the size of $List$, in $r$,  grows monotonically,  hence  $w_{\pi}$ invoked $\pi$ after $\phi$ completes hence $tag(\pi)$ is at least as high as $tag(\phi)$, so $\pi\not \prec \phi$.
	        %
	         \item [$(d)$] $\phi$ is a \rot{}, $\pi$ is a \wot{}  invoked by reader $r$ and $w_{\pi}$, respectively.
	         This case is simple because new values are added to $List$  only  by writers, and $tag(\pi)$ 
	         is larger than the tag of $\phi$ and hence   $\pi\not \prec \phi$. 
	\end{enumerate}
	
	\emph{P3:} This is clear by the fact that any \wot{} always creates a unique tag and all tags are totally ordered since they all belong to $\mathbb{N}$
	
	\emph{P4:} Consider a \rot{} $\rho$ as $READ(o_{i_1}, o_{i_2}, \cdots, o_{i_q})$, in $\beta$. 
Let the returned value from $\rho$ be $\mathbf{v} \equiv $$(v_{i_1}, v_{i_2}, \cdots, v_{i_q})$ such that 
$1 \leq {i_1} <  {i_2} <  \cdots <  {i_q} \leq k$, where value  $v_{i_j}$ corresponds to $o_{i_j}$. 
	Suppose $tag(\rho) \in \mathbb{N}$ was created during some \wot{}, say $\phi$, i.e., $\phi$ is the \wot{} that 
	added the elements in index $(tag(\rho)-1)$ of $List$. Note that element in index $0$ contains the initial value.
	%because \rots{} do not generate new tags as they do not add any new item to the $Vals$ of any server.
	 Now we consider two cases:
	 
	\emph{Case $tag(\rho) = 1$.} We  know that it corresponds the initial default value $v_i^0$ at each sub-object $o_i$, and this equates to $\rho$ returning the default initial value for each sub-object.
	 %
	% Therefore, $tag(\rho) = tag(\phi)$. 
	 
	 \emph{Case $tag(\rho) > 1$.} Then we argue that there exists no \wot{}, say $\pi$, that updated object $o_{i_j}$,   in $\beta$, such that,  $\pi \neq \phi$ and $\rho$ returns values written by $\pi$ and $\phi \prec \pi \prec \rho$. Suppose we assume the 	contrary, which means $tag(\phi) < tag(\pi) < tag(\rho)$. The latter implies $tag(\phi)  = tag(\pi)$ which is not possible because 
	this contradicts the fact that for any two distinct \wots{} $tag(\phi) \neq tag(\pi)$  in any execution of   $A$.
	
	\noindent{\emph{\underline{N property:}}}  By inspection of algorithm $A$ for the  response steps  of the servers to the reader.
	
	\noindent{\emph{\underline{O property:}}} By inspection of the  {\readValue} phase: it consists of one round of communication between the reader and the servers, where the servers send only one version of the value of the object it maintains.
	
	\noindent{\emph{\underline{W property:}}}  By inspection of the \wot{} steps, and  and  that writers always get to complete the transactions they invoke.
	%Finally, the liveness property of \wots{} can be realized by inspecting the steps of  algorithm $A$.
	\end{proof}
}

 %Note that the above theorem still holds even if any writer crashes.

