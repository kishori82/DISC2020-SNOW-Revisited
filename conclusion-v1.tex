We revisited the SNOW Theorem and when it is possible for READ transactions to have the same latency as simple reads.
We provided a new and more rigorous proof of the original result.
We also closed several open questions that were either explicitly posed by the original work or that emerged from our careful analysis.
We found that READ transactions can match the latency of simple reads when client-to-client communication is allowed in MWSR setting.
We found that they cannot and must have higher worst-case latency when client-to-client communication is disallowed or there are at least two readers.
We also presented the first algorithms that provide bounded worst-case latency for read-only transactions in strictly serializable systems with 
WRITE transactions.

\remove{
There are, however, several remaining important questions.
The first three questions focus on bounds for read-only transactions in strictly serializable systems with WRITE transactions.
First, is it possible to return exactly one version, always complete in two rounds, but that also can optimistically complete after one round?
Second, is it possible to always complete in one round and return a bound on number of versions that is less than the number of concurrent 
WRITEs?
Third, is there a way to enhance our model such that blocking could be bounded, and, if so, what bounds on blocking are possible?
Another open question is whether our new algorithms for READ transactions could be adapted to work in systems with READ-WRITE 
transactions that are more powerful than WRITE transactions? We conjecture they can be.

Our final open questions focus on ever more practical settings. 
%Each of our algorithms for bounding latency uses a 
coordinator, which in a real system would eventually become bottlenecked and restrict scaling to arbitrary sizes.
Is it possible to design algorithms that match our bounds on 
latency without relying on a centralized component and thus could scale arbitrarily?
Can one design a coordinator that can scale to arbitrary throughputs?
}