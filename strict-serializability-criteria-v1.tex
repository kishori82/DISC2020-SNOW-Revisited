%\section{Condition for proving strict serializability}\label{sc_condition_app}
In this section, we derive a useful property for  executions of  algorithms that implement  objects of data type $\mathcal{O}_T$  that will later  help us  show the \emph{strict serializability} property (S property) of algorithms presented in later sections.  

Although the strict serializability property in transaction-processing systems is a well-studied topic,  
the specific setting considered  in this paper is much simpler. Therefore, this allows  us to derive
 simpler conditions to prove the safety of these algorithms.  A wide range of transaction types and 
 transaction processing systems are considered in the literature. For example, in ~\cite{Papadimitriou79}, Papadimitriou   
defined the strict serializability conditions as a part of developing a  theory for analyzing transaction 
processing systems. In this work, each transaction $T$ consists of a set of write operations $W$, at 
individual objects, and a set of read operations $R$  from individual objects, where  the operations in 
$W$  must complete before the operations in $R$ execute.  
Other types of transaction processing systems allow nested transactions
%~\cite{Liskov1988, Lynch1994, Gary1993},
\cite{Gary1993},
 where the transactions may contain sub-transactions~\cite{Bernstein:1987} which may 
further contain a mix of read or write operations, or even child-transactions. 
In most transaction processing systems considered in the literature,  transactions can be 
\emph{aborted} so as  to handle  failed  transactions. As a result, the serializability theories are 
developed  while  considering the presence of  aborts. 
However, in our system, we do not consider any abort, nor any  client or  server failures.  A transaction 
in our system is either a set of independent writes or a set of reads %(see Section~\ref{datatype_app}) 
with 
all the reads or writes in a transaction operating on different  objects. Such simplifications allow us to 
formulate an equivalent condition for the execution of an algorithm  to prove the S  
property of such algorithms while  implementing an object of data type $\mathcal{O}_T$.

\sloppy We note that an execution of a variable of type $\mathcal{O}_T$ is a finite sequence 
$\mathbf{v}_0, INV_1, RESP_1,$ $\mathbf{v}_1, INV_2,$ $ RESP_2, \mathbf{v}_2, \cdots, \mathbf{v}_r$ or an infinite sequence 
$\mathbf{v}_0, INV_1, RESP_1, \mathbf{v}_1, INV_2, RESP_2, \mathbf{v}_2, \ldots$, where $INV$'s and 
$RESP$'s are invocations and 
responses, respectively. The $\mathbf{v}_i$s are tuples  of the the form $(v_{1}, v_{2}, \cdots, v_{k}) \in \Pi_{i=1}^{q} V_{i}$, that 
corresponds to the latest values stored across the objects $o_1, o_2\cdots o_k$, and the values in  $\mathbf{v}_0$ are  the initial 
values of the objects. Any adjacent quadruple such as  $\mathbf{v}_i, INV_{i+1}, RESP_{i+1},$ $\mathbf{v}_{i+1}$  is consistent with the $f$ 
function for an object of type $\mathcal{O}_T$ (see  Section~\ref{datatype_app}) . Now, the safety property of such an 
object is a trace that describes the 
correct response to a sequence of $INV$s when all the transactions are executed sequentially. The \emph{strict serializability} of 
$\mathcal{O}_T$ says that each trace produced by an execution of $\mathcal{O}_T$ with concurrent transactions appears as some trace of $\mathcal{O}_T$. We describe this below in more detail. 
% For each completed transaction $\pi$ we assume 
%$INV$ and $RESP$ occur consecutively at $\pi_*$. Also, for each $\pi \in \Phi$ put $INV$ at $\pi_*$ and remove all invocations.

\begin{definition}[Strict-serializability] Let us consider an execution $\beta$ of an object of type $\mathcal{O}_T$, such that the invocations of any transaction at any 
client respects the well-formedness property.  Let $\Pi$ denote the set of complete transactions in $\beta$ then we say $\beta$ satisfies the strict-serializability property for 
$\mathcal{O}_T$ if the following are possible:
\begin{enumerate}
\item[$(i)$] For every complete {\sc read} or {\sc write} transaction $\pi$ we insert a point (serialization point) $\pi_*$ between the actions   $\INV{\pi}$ and $\RESP{\pi}$.
\item[$(ii)$] We select a set $\Phi$ of incomplete transactions in $\beta$ such that for 
each $\pi \in \Phi$ we   select a response $\RESP{\pi}$.
\item[ $(iii)$] For each $\pi \in \Phi$ we insert $\pi_*$ somewhere after $\INV{\pi}$ in $\beta$, and remove the $INV$ for the rest of the incomplete transactions in $\beta$.
\item[$(iv)$] If we assume for each $\pi \in \Pi \cup \Phi$ both $\INV{\pi}$ and $\RESP{\pi}$ to occur consecutively at $\pi_*$, with the interval of the transaction shrunk to $\pi_*$,   then the sequence of transactions in this  new trace  is a trace of an object of data type  $\mathcal{O}_T$.
\end{enumerate}
\end{definition}
Now, we consider any automaton $\mathcal{B}$ that implements an object of  type $\mathcal{O}_T$, and prove a result that serves us an  equivalent condition for proving  the strict serializability property of $\mathcal{B}$. Any trace property $P$ of an automaton  is a \emph{safety property}  if the set of executions in $P$ is non-empty; \emph{prefix-closed}, meaning any prefix of an execution in $P$ is also in $P$; and \emph{limit-closed}, i.e., if $\beta_1$, $\beta_2, \cdots$ is any infinite sequence of executions in $P$ is such that $\beta_i$ is prefix of $\beta_{i+1}$ for any $i$, then the limit $\beta$ of the sequence of executions $\{\beta_i\}_{i=0}^{\infty}$  is also in $P$. From Theorem~13.1 in~\cite{Lynch1996}, we know that the trace property, which we denote by $P_{SC}$, of 
any well-formed execution of $\mathcal{B}$ that satisfies the strict-serializability property is a safety property. Moreover, from 
Lemma 13.10 in~\cite{Lynch1996} we can deduce that  if  every execution of $\mathcal{B}$ that is well-formed and failure-free, and also contains no incomplete transactions,  satisfies $P_{SC}$, then any well-formed  execution of $\mathcal{B}$ that can possibly have incomplete transactions is also in $P_{SC}$. 
Therefore, in the following lemma,  which gives us an equivalent condition for the strict serializability property of an execution $\beta$, we consider only executions without any incomplete transactions.
The lemma is proved in a manner similar to Lemma 13.16 in~\cite{Lynch1996}, for atomicity guarantee of a single multi-reader multi-writer object.


\begin{lemma}\label{lem:equivalence_app}
Let $\beta$ be an execution (finite or infinite) of  an  automaton $\mathcal{B}$ that implements an object of type $\mathcal{O}_T$, which consists of  a set of $k$ sub-objects.
%, with strictly serializability  property.
%such that for each transaction that are only finitely many transactions in $\beta$  precessing it with respect to (w.r.t.) $\prec$.
 Suppose all clients in $\beta$ behave in an well-formed
manner. Suppose $\beta$ contains no incomplete transactions and let  $\Pi$ be the set of transactions in $\beta$. Suppose there exists an irreflexive partial ordering ($\prec$)  among  the transactions
 in $\Pi$, such that,
 \begin{enumerate}
  \item [$P1$] For any transaction $\pi \in \Pi$ there are only a finite number of transactions $\phi \in \Pi$ such that 
    $\phi \prec \pi$;
 \item [$P2$] If the response event for $\pi$ precedes the invocation event for $\phi$ in $\beta$, then it cannot be that $\phi \prec \pi$;
 \item [$P3$]If $\pi$ is a {\sc write} transaction  in $\Pi$ and $\phi$ is any transaction in $\Pi$, then either $\pi \prec \phi$ or $\phi \prec \pi$; and 
 \item [$P4$] \sloppy A tuple 
 $ \mathbf{v} \equiv $$(v_{i_1}, v_{i_2}, \cdots, v_{i_q})$  
returned by a
$READ(o_{i_1}, o_{i_2}, \cdots, o_{i_q})$,  where $q$ is any positive integer, $1 \leq q \leq k$, 
 is such that  $v_{i_j}$  $j \in \{1, \cdots, q\}$ is written in $\beta$ by the last preceding (w.r.t. $\prec$)  {\sc write} transaction that contains a $\writeop{o_{i_j}}{*}$,
 or the initial value $v_{i_j}^0$ if no such {\sc write} exists in $\beta$. 
 \end{enumerate}
Then execution $\beta$ is strictly serializable.
\end{lemma}

\begin{proof}We discuss how to insert a serialization point $*_{\pi}$ in $\beta$ for every transaction $\pi \in \Pi$. 
First, we add  $*_{\pi}$ immediately after the latest of the invocations of $\pi$ or $\phi \in \Pi$ such that $\phi \prec \pi$.
We want to stress that any complete operation $\pi$ in $\Pi$  refers to an operation, with the invocation and response events, whereas
$\pi_*$ refers to a point in the executions.
Note that according to condition $P1$ for $\pi$ there are only finite number of such  invocations in
 $\beta$, therefore, $\pi_*$ is well-defined for any $\pi \in \Pi$. Now, since the order of the invocation events of the 
transactions in $\Pi$ are already defined, therefore,  the order of the corresponding set of serialization points are 
well-defined, except for the case when more than one serialization points are placed immediately after an 
invocation.
 In the case such multiple  serialization points corresponding to an invocation we order  these 
serialization points  in accordance with the $\prec$  relation of  the underlying transactions.

Next, we show that for any pair of transactions $\phi$, $\pi \in \Pi$ if $\phi \prec \pi$ then in $\beta$ we have $*_{\phi}$ precedes $*_{\pi}$.
Suppose $\phi \prec \pi$. By construction,  each of $\pi_*$ and $\phi_*$ appear immediately after 
some invocation, in $\beta$,  of some transaction in $\Pi$. If both $\pi_*$ and $\phi_*$ appear immediately  after 
the same invocation, then since $\phi \prec \pi$,  by construction of $\pi_*$,  the serialization point $\pi_*$ appears in $\beta$ after 
$\phi_*$.  Also, if the invocations after which $\pi_*$ and $\phi_*$ appear are distinct,  then by 
construction of $\pi_*$ the serialization point  $\pi_*$ appear after $\phi_*$ in $\beta$ since $\phi \prec \pi$.

 %Observe that, by the property of a partial order relation,  for any $\psi \in \Pi$ if $\psi \prec \phi$ then $\psi \prec \pi$. Therefore, according to the above scheme for ordering serialization point for operation we have $*_{\phi}$ precedes $*_{\pi}$. 

Next we argue that each $*_{\pi}$  serialization point  for any $\pi \in \Pi$ is placed between the 
invocation $\INV{\pi}$ and responses $\RESP{\pi}$. By construction, $*_{\pi}$ is after $\INV{\pi}$ in $\beta$. To 
show that $*_{\pi}$ is before $\RESP{\pi}$ for the sake   of contradiction assume that  $*_{\pi}$  
appears after $\RESP{\pi}$. By construction, $*_{\pi}$ must be after $\INV{\psi}$ for some 
$\psi \in \Pi$ and $\psi \neq \pi$, then by the condition of construction of $\pi_*$ we have $\psi \prec 
\pi$. But from above 
$\INV{\psi}$ occurs after $\RESP{\pi}$, i.e., $\pi$ completes before $\psi$ is invoked which means, by 
property $P2$,  we cannot have $\psi \prec \pi$, which is a contradiction.

Next, we show that 
if we were to shrink  the transactions intervals to  their corresponding serialization points, the 
 resulting  trace would be a trace of the underlying data type $\mathcal{O}_T$. 
In other words, we show any {\sc read}
%a {\sc read}s in $\beta$ must  return the values $(v_1, v_2, \cdots v_k)$, for every  $i \in [k]$ $v_i$.
$READ(o_{i_1}, o_{i_2}, \cdots, o_{i_q})$
%,  where $o_{i_1}, o_{i_2}, \cdots, o_{i_q}$ is a set of distinct objects,  $q$ is any positive integer, $1 \leq q \leq k$, which upon completion  
returns the  values   $(v_{i_1}, v_{i_2}, \cdots, v_{i_q})$, such that each value
$v_{i_j}$, $j \in [q]$,  was written by the immediately preceding  (w.r.t. the serialization points)  {\sc 
write} that contained  $\writeop{o_{i_j}}{v_{i_j}}$ or the initial values if no such previous {\sc write} exists.
%
Let us denote the set of {\sc write}s that precedes (w.r.t. $\prec$) $\pi$ by $\Pi_W^{\prec\pi}$, i.e., 
$\phi \in \Pi^{\prec\pi}_W$  $\phi$ is a write and $\phi \prec \pi$.  By property $P3$, all transactions in 
$\Pi_{W}^{\prec\pi}$ are totally-ordered.  By property $P4$,  $v_{i_j}$ must be the value updated by
 the most recent {\sc write} in $\Pi_W^{\prec\pi}$.  Since the total order of serialization points are  consistent with 
 $\prec$ and hence the $v_{i_j}$ corresponds to the write operation of a {\sc write} transaction with 
  the most recent serialization point and  contains a operation of  type $\writeop{o_{i_j}}{*}$.
\end{proof}

