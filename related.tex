In this work, the new impossibility results built upon the SNOW Theorem are philosophically similar to other impossibility results---such as FLP~\cite{Fischer:pds1983} and CAP~\cite{Brewer:pdc2000, Gilbert:sigact2002}---in that they help system designers avoid wasting effort in trying to achieve the impossible. That is, the SNOW Theorem identifies a boundary in the design space of \rots{}, beyond which no algorithms can possibly exist. By revisiting SNOW, our work makes this boundary more precise. 


The SNOW Theorem and our revisitation in this work are the first set of results on the performance-guarantee tradeoff due to sharding with a focus on \rots{}. Most prior results related to sharding focus on the complexity-robustness tradeoff~\cite{Guerraoui:pods2017}, which states that transaction protocols become more complex when there are fewer restrictions on how servers fail. In contrast, our results apply to failure-free cases, which are arguably the most frequent in practice~\cite{Guerraoui:pods2017}, and scenarios where machines can fail.

Besides making more precise conditions for the impossibility results, this work also
introduces novel possible algorithms.
Before our work, no existing algorithms achieved bounded latency with the strongest guarantees. They either required an unbounded number of round trips with single version~\cite{Lloyd:nsdi2013, Wei:sosp2015, Lee:sosp2015, Aguilera:sosp2007}, blocked read operations~\cite{Corbett:osdi2012, SNOW2016}, or settled for weaker guarantees~\cite{Corbett:osdi2012, Aguilera:sosp2015, Bailis:sigmod2014, Lloyd:sosp2011}.


%\wl{We should discuss the work on atomic commit here:
%``Most prior results in the sharding dimension are related to atomic commit, which are not applicable to \rots{} because it is always safe for them to ``commit'' because they do not update data.
%\hl{I think this might raise confusion since read transactions do sometimes abort. people might question why we say it's always safe to commit? I think we could even take out this sentence.}
%''
%}
